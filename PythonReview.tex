\documentclass[11pt]{article}
\pdfpagewidth 8.5in
\pdfpageheight 11in

\setlength\topmargin{0in}
\setlength\headheight{0in}
\setlength\headsep{0.25in}
\setlength\textheight{7.7in}
\setlength\textwidth{6.5in}
\setlength\oddsidemargin{0in}
\setlength\evensidemargin{0in}
\setlength\parindent{0.25in}
\setlength\parskip{0.5in} 
\usepackage{verbatim}
%\usepackage{geometry}                % See geometry.pdf to learn the layout options. There are lots.
%\geometry{letterpaper}                   % ... or a4paper or a5paper or ... 
%\geometry{landscape}                % Activate for for rotated page geometry
\usepackage[parfill]{parskip}    % Activate to begin paragraphs with an empty line rather than an indent
\usepackage{graphicx}
\usepackage{amssymb,amsmath}
\usepackage{epstopdf}
\DeclareGraphicsRule{.tif}{png}{.png}{`convert #1 `dirname #1`/`basename #1 .tif`.png}
\usepackage{fullpage}
\usepackage{mdwlist}
\usepackage{enumitem}
\setdescription{noitemsep,topsep=0pt,parsep=0pt,partopsep=0pt}

\title{Python and Command Line Review}
\author{Emily Josephs \& Nancy Chen}
%\date{}                                           % Activate to display a given date or no date

\begin{document}
\maketitle
%\section{}
%\subsection{}


This is not a quiz! Please work through the questions. If you get stuck you can try actually running the scripts to see how they work. \\
\textsl{Typeset conventions}: All code or code-like text is written in \texttt{constant-width font}.\\
\begin{enumerate}
\section*{Unix}
\item Briefly define what the following unix commands will do.
  \begin{itemize}
  \item \texttt{cd /home/mydirectory}\\
  \item \texttt{cd ..}\\
  \item \texttt{cp}\\
  \item \texttt{less}\\
  \item \texttt{head}\\
  \end{itemize}

\item How do you save a file in vim? \\
  \\


\section*{Data types}

\item Data can be stored as integers, floats, strings, or lists. Write the data type of each variable listed below.
\begin{itemize}
\item \texttt{'tomato'}
  \item \texttt{56}
  \item \texttt{72.2}
  \item \texttt{['apple','pear','plum']}
    \item \texttt{[1,2,3,4,5]}\\
\end{itemize}

\item \texttt{myFruit = ['apple','pear','plum']}. What does \texttt{myFruit[0]} refer to?\\
  \\
\item What would the output of \texttt{len(myFruit)} be?\\
  \\

\item Write down the code would you use to add \texttt{'tomato'} to the list \texttt{myFruit}.\\
\\
\\
\\
\\
\section*{If/Else and For Loops}
\item Look at the following script:
\begin{verbatim}
    myVeg = ['carrot','beet','bok choy']
    if 'cabbage' in myVeg:
        print('yay I can make cole slaw!')
    else:
        print('I need to go to the store')
\end{verbatim}

What will be printed out if you run the script?\\
\\
\\
\\
\item Look at the following script:
\begin{verbatim}
    myVeg = ['carrot', 'beet', 'bok choy']
    for veg in myVeg:
        print(myVeg[0])
\end{verbatim}
What will be printed if you run the script?\\
\\
\\
\\
\\

\item Look at the following line:
\begin{verbatim}
    myCodons = ["atg","tac","ttt","tcc"]
\end{verbatim}
Write a short script that will tell you if the codon \texttt{`atg'} is in the list \texttt{myCodons}\\ 
\\
\\
\\
\\
\\
\\
\\
\\
\\
\\

\item
  \texttt{codonTable[``atg'']} will return the amino acid coded by the codon \texttt{`atg'}. Write a script that will print out the amino acids coded by the four codons in \texttt{myCodons}. You will likely need to use a \texttt{for} loop.\\
  \\
  \\
  \\
  \\
  \\
  \\
  \\
  \\
  \\
  \\
\item Open a new python file in vim and type the following code:
\begin{verbatim}
myRange = range(0,10)
print(myRange)
\end{verbatim}
Write down what you think the \texttt{range} function does. What data type is the output?\\
\\
\\
\\
\\
\\
Edit the script to print out all the numbers between 20 and 30.
\\
\end{enumerate}
\end{document}  
